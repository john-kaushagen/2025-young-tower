\begin{theorem}
    Suppose $P_n = O(n^{-\alpha}\ell_1(n))$ and $u_n = O(n^{-\beta}\ell_2(n))$ where $1 \leq \alpha < \beta$, and $\ell_1$ and $\ell_2$ are two slowly varying functions.
    If $\alpha = 1$, then
    \[
        \left|a_n - \frac{1}{\mu}\right| = \frac{1}{\mu^2}V_n + o\left(\ell_1^*(n)\right),
    \]
    and if $\alpha > 1$, then
    \[
        \left|a_n - \frac{1}{\mu}\right| = \frac{1}{\mu^2}V_n + o(n^{-\alpha+1}\ell_1(n))
    \]
    where
    \[
        V_n = \begin{cases}
            O\left(\ell^*_1(n)\right)
                & \text{if}\ \alpha = 1,\\
            O\left(n^{-\alpha + 1} \ell_1(n)\right)
                & \text{if}\ \alpha > 1.
        \end{cases}
    \]
\end{theorem}
\begin{proof}
    Recall the representation of $a_n$ from Lemma \ref{thm:a_n-decomp}:
    \[
       a_n = \frac{1}{\mu}
           - \frac{1}{\mu}\sum_{k=n+1}^\infty u_k
           + \frac{1}{\mu^2} \sum_{k=0}^n u_k Q_{n-k}
           + R_n.
    \]
    Since $u_n =  O(n^{-\beta}\ell_2(n))$ and $\beta > 1$ by assumption,
    \[
        U_n = \sum_{k=n+1}^\infty u_k = O(n^{-\beta+1}\ell_2(n))
    \]
    by Lemma \ref{lemma:rogozin-slow-var-tail}.

    We consider the terms $V_n$ and $R_n$ in two separate cases: when $\alpha = 1$ and when $\alpha > 1$.

    \case{$\alpha = 1$}
    In this case, $P_n = O(n^{-1}\ell_1(n))$,
    so
    \[
        Q_n = \sum_{k=n+1}^\infty P_k = O\left(\sum_{k=n+1}^\infty k^{-1}\ell_1(k)\right) = O(\ell_1^*(n))
    \]
    by Lemma \ref{lemma:rogozin-slow-var-tail}.
    Combing this with the fact that $u_n = O(n^{-\beta}\ell_2(n))$ where $\beta > 1$,
    we can use Lemma \ref{lemma:convolution-domination} to see that
    \[
        V_n = \sum_{k=0}^n u_k Q_{n-k} = O(\ell_1^*(n)).
    \]

    Lastly, we consider the terms $R_n$.
    From Rogozin's result in Theorem \ref{theorem:rogozin},
    we have $r_n(1) = O(\ell_1^*(n))$.
    Therefore,
    \[
        R_n = \sum_{k=0}^n u_k r_{n-k}O(\ell_1^*(n))
    \]
    again by Lemma \ref{lemma:convolution-domination}.

    Combining these two results,
    we see that $V_n = O(\ell_1^*(n))$, while $u_n, R_n = o(\ell_1^*(n))$.
    Thus we have
    \[
        \left|a_n - \frac{1}{\mu}\right| = \frac{1}{\mu^2}V_n + o\left(\ell_1^*(n)\right)
    \]
    when $\alpha = 1$.

    \case{$\alpha > 1$}
    When $\alpha > 1$, we have $Q_n = O(n^{-\alpha+1}\ell_1(n))$ by Lemma \ref{lemma:rogozin-slow-var-tail},
    and $r_n = O(n^{-\alpha}\ell_1(n))$ by Theorem \ref{theorem:rogozin}.
    These two bounds, combined with the fact that $u_n = O(n^{-\beta}\ell_2(n))$, give us
    $V_n = O(n^{-\alpha+1}\ell_1(n))$ and $R_n = O(n^{-\alpha}\ell_1(n))$ by Lemma \ref{lemma:convolution-domination}.
    
    Since $\beta > \alpha$,
    we have that both $u_n = o(n^{-\alpha+1}\ell_1(n))$ and $R_n = o(n^{-\alpha+1}\ell_1(n))$.
    Thus the dominating term is $V_n$, and we have
    \[
        \left|a_n - \frac{1}{\mu}\right| = \frac{1}{\mu^2}V_n + o\left(n^{-\alpha+1}\ell_1(n)\right)
    \]
    proving the result.
\end{proof}

The next theorem gives the result when then delay terms $u_n$ dominate the renewal terms $P_n$.
\begin{theorem}
    \label{theorem:main-beta-lt-alpha}
    Suppose $P_n = O(n^{-\alpha}\ell_1(n))$
    and $u_n = O(n^{-\beta}\ell_2(n))$
    where $1 \leq \beta < \alpha$,
    and $\ell_1$ and $\ell_2$ are two slowly-varying functions.
    Then we have the following bounds: if $\beta = 1$, then
    \[
        \left|a_n - \frac{1}{\mu}\right| = \frac{1}{\mu}U_n + o\left(\sum_{k=1}^n k^{-1}\ell_2(k)\right)
    \]
    and if $\beta > 1$, then
    \[
        \left|a_n - \frac{1}{\mu}\right| = \frac{1}{\mu} U_n + o(n^{-\beta+1}\ell_2(n))
    \]
    where
    \[
        U_n = \begin{cases}
            O\left(\sum_{k=0}^n k^{-1}\ell_1(k)\right)
                & \text{if}\ \beta = 1,\\
            O\left(n^{-\alpha + 1} \ell_1(n)\right)
                & \text{if}\ \beta > 1.
        \end{cases}
    \]
\end{theorem}
\begin{proof}
    We recall that the sequences $Q_n$ and $U_n$ are defined as
    \begin{equation}
        Q_n = \sum_{k=n+1}^\infty P_k,
        \text{ and }
        U_n = \sum_{k=n+1}^\infty u_k.
    \end{equation}
    Since $\alpha > 1$, Lemma \ref{lem:sv-tail-rog} gives us the asymptotic bound $Q_n = O(n^{-\alpha + 1}\ell_1(n))$. From the same lemma, we also get the bound
    \begin{equation}
        U_n = \begin{cases}
            O\left(\sum_{k=n}^\infty k^{-1}\ell_2(k)\right) & \text{if } \beta=1,\\
            O(n^{-\beta+1}\ell_2(n)) & \text{if } \beta > 1.
        \end{cases}
    \end{equation}

    Our definition of $V_n$ is 
    \[
        V_n = \sum_{k=0}^n u_k Q_{n-k},
    \]
    so the fact that $Q_n = O(n^{-\alpha+1}\ell_1(n))$ and $u_n = O(n^{-\beta}\ell_2(n))$, we get by Lemma \ref{lemma:convolution-domination} that $V_n = O(n^{-\min(\beta, \alpha-1)}\max(\ell_1(n),\ell_2(n))$. The exponent $-\min(\beta, \alpha-1)$ in the bounds of $V_n$ is strictly greater than the exponent $-\beta+1$ in the bounds of $U_n$, so we get that $V_n = o(n^{-\beta+1}\ell_2(n))$.

    The argument for the term $R_n$ is similar. We know that
    \[
        R_n = \sum_{k=0}^n u_k r_{n-k}
    \]
    by definition. Furthermore, from Rogozin's original theorem (Theorem \ref{theorem:rogozin}, we know that $r_{n-k}$ is a smaller order that $Q_{n-k}$, and so by Lemma \ref{lemma:convolution-domination}, the order of $R_n$ can be no larger than the order of $V_n$.

    From this, we see the dominating order is that which bounds $U_n$. Thus when $1 \leq \beta < \alpha$ we have
    \[
        \left|a_n - \frac{1}{\mu}\right| = \frac{1}{\mu} U_n + o\left(\sum_{k=n}^\infty k^{-1}\ell_2(k)\right)
    \]
    when $\beta = 1$, and
    \[
    \left|a_n - \frac{1}{\mu}\right| = \frac{1}{\mu} U_n + o(n^{-\beta+1}\ell_2(n))
    \]
    when $\beta > 1$.
\end{proof}
